%%%%%%%%%%%%%%%%% PREAMBLE %%%%%%%%%%%%%%%%%%%%%%%%%%%%
%Change the font size of your document - 10pt, 12.1pt, etc.
\documentclass[article,9pt,oneside]{article}
\usepackage[utf8]{inputenc}
\usepackage{setspace}
\usepackage{hyperref}
\usepackage{tabularx}

\usepackage{graphicx}
\graphicspath{ {images/}} %upload your signature to this file
%Change the margins to fit your CV/resume content
\usepackage[left=1in, right=1in, bottom=0.75in, top=0.75in]{geometry}

\setlength\parindent{0pt}

%Changes the page numbers - {arabic}=arabic numerals, {gobble}=no page numbers, {roman}=Roman numerals
\pagenumbering{gobble}
\begin{document}

\section*{Parameter guide}

\begin{center}
	\begin{tabular}{|p{4cm}|p{1.5cm}|p{3cm}|p{8cm}|} \hline
		\textbf{Parameter name} & \textbf{ Units } & \textbf{Standard value} & \textbf{description} \\ \hline 
		  r1 & mm & 70.00 & Radius of inner-inner (sub-super) fiber-layer\\ \hline
		  r2 & mm & 70.86 &Radius of outer-inner  fiber-layer\\  \hline
		  r3 & mm & 98.00 & Radius of inner-outer fiber-layer\\ \hline
		  r4 & mm & 98.86 &  Radius of outer-outer fiber-layer\\  \hline
		  IL1\_min&\#& 0 & Fiber number of leftmost fiber of inner-inner fiber-layer \\  \hline
		  IL1\_max&\#& 197 &Fiber number of rightmost fiber of inner-inner fiber-layer  \\  \hline
		  IL2\_min&\#& 200 &Fiber number of leftmost fiber of outer-inner fiber-layer  \\ \hline
		  IL2\_max&\#& 398 & Fiber number of rightmost fiber of outer-inner fiber-layer \\ \hline
		  IL3\_min&\#& 400 & Fiber number of leftmost fiber of inner-outer fiber-layer \\ \hline
		  IL3\_max&\#& 597 &Fiber number of rightmost fiber of inner-outer fiber-layer  \\ \hline
 		  IL4\_min&\#& 600 &Fiber number of leftmost fiber of outer-outer fiber-layer  \\ \hline
		  IL4\_max&\#& 798 &Fiber number of rightmost fiber of outer-outer fiber-layer   \\ \hline
		  last\_fiber\_position&mm& 118.8 & the position of the outermost fiber at the inner sublayers of both superlayers from the vertical central axis \\ \hline
		  fiber\_diameter&mm& 1& The diamater of a single Kuraray SCSF-78M plastic scintillating fiber.\\ \hline
		  inter\_fiber\_dist&mm& 1.2& The distance between the central axis going through two mounted fibers\\ \hline
		  layer\_offset&mm& 0.6 & The horizontal distance between two neighboring fibres, where one is positioned in the upper sublayer and the other in the lower sublayer of a single superlayer   \\ \hline
		  edge &\#& 700 & The number of simultaneously active fibres indicating the arrival of the particle injection. This number sets the initial time cut on the total data array. \\ \hline
		  N\_fibers&\#& 798& Note: Real amount of fibers are 794, but no fibers with index 198,199, 399, 598 or 599. \\ \hline
		  Time\_resolution&ns& 5& The time resolution of the electronic readout system, currently at 5ns.  \\ \hline
		  max\_travel\_time &ns&5& The max traveling time for a particle to traverse the detector, note that this one will have to be specified for each particle\\ \hline 
		  Track\_radius&mm&1.6& The radious around a point for which to search for clustering \\ \hline
		  pion\_travel\_time &m/ns&0.1& The traveling velocity of a pion\\ \hline
		  edge\_buffer &\#&0& A parameter to be set when some data before the rising edge needs to be acquired.  \\ \hline
		  start\_time&ns&2000& The time after the rising edge of which to begin sampling the tail data \\ \hline
		  stop\_time&ns&20000& The time after the rising edge of which to stop sampling tail data\\ \hline
		  min\_time&ns&0& minimum sampling time\\ \hline
		  max\_time&ns&30000& maximum sampling time\\ \hline
	\end{tabular}
\end{center}


\end{document}
